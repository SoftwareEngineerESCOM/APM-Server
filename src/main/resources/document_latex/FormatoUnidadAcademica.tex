\documentclass[10pt]{article}
\usepackage[utf8]{inputenc}
\usepackage[spanish]{babel}
\decimalpoint
\usepackage{amsmath}
\usepackage{caption}
\usepackage{amsthm}
\usepackage{amssymb}
\usepackage{graphicx}
\usepackage[left=1.2cm,top=1.2cm,right=2.6cm,bottom=1.7cm]{geometry} 
\usepackage{fancyhdr}
\usepackage[inline]{enumitem}
\usepackage{float}
\usepackage{cancel}
\usepackage{bigints}
\usepackage{color}
\usepackage{xcolor}
\usepackage{listingsutf8}
\usepackage{algorithm}
\usepackage{tocloft}
\usepackage[none]{hyphenat}
\usepackage{graphicx}
\usepackage{grffile}
\usepackage{tabularx}
\usepackage[nottoc,notlot,notlof]{tocbibind}
\usepackage{times}
\usepackage{color}
\usepackage{lastpage}
\usepackage{enumerate}% http://ctan.org/pkg/enumerate
\definecolor{gray97}{gray}{.97}
\definecolor{gray75}{gray}{.75}
\definecolor{gray45}{gray}{.45}
\renewcommand{\cftsecleader}{\cftdotfill{\cftdotsep}}
\pagestyle{empty}
\definecolor{pblue}{rgb}{0.13,0.13,1}
\definecolor{pgreen}{rgb}{0,0.5,0}
\definecolor{pred}{rgb}{0.9,0,0}
\definecolor{pgrey}{rgb}{0.46,0.45,0.48}
\lstset{tabsize=1}
\usepackage{ragged2e}
\usepackage{listings}
\lstset{ frame=Ltb,
framerule=0pt,
aboveskip=0.5cm,
framextopmargin=3pt,
framexbottommargin=3pt,
framexleftmargin=0.4cm,
framesep=0pt,
rulesep=.4pt,
backgroundcolor=\color{gray97},
rulesepcolor=\color{black},
stringstyle=\ttfamily,
showstringspaces = false,
basicstyle=\small\ttfamily,
commentstyle=\color{gray45},
keywordstyle=\bfseries,
numbers=left,
numbersep=15pt,
numberstyle=\tiny,
numberfirstline = false,
breaklines=true,
}
% minimizar fragmentado de listados
\lstnewenvironment{listing}[1][]
{\lstset{#1}\pagebreak[0]}{\pagebreak[0]}
\lstdefinestyle{consola}
{basicstyle=\scriptsize\bf\ttfamily,
backgroundcolor=\color{gray75},
}
\lstdefinestyle{Java}
{language=Java,
}
%%%%%%%%%%%%%%%%%%%%%
\lstdefinestyle{customc}{
  belowcaptionskip=1\baselineskip,
  breaklines=true,
  frame=L,
  xleftmargin=\parindent,
  language=C,
  showstringspaces=false,
  basicstyle=\footnotesize\ttfamily,
  keywordstyle=\bfseries\color{green!40!black},
  commentstyle=\itshape\color{purple!40!black},
  identifierstyle=\color{blue},
  stringstyle=\color{orange},
}
\lstdefinestyle{customasm}{
  belowcaptionskip=1\baselineskip,
  frame=L,
  xleftmargin=\parindent,
  language=[x86masm]Assembler,
  basicstyle=\footnotesize\ttfamily,
  commentstyle=\itshape\color{purple!40!black},
}
\lstset{escapechar=@,style=customc}
\usepackage{cmbright}
\usepackage{multirow}
\newcommand\tab[1][1cm]{\hspace*{#1}}
% Font
%Permite crear columnas en el documento
\usepackage{multicol} 
\usepackage{color}
\usepackage{comment}
\newcommand{\tabitem}{~~\llap{\textbullet}~~}
\newcommand{\subtabitem}{~~~~\llap{\textbullet}~~}
\usepackage{longtable}
% Ayuda para el formato de las tablas
\usepackage{array}
% Se declara un nuevo tipo de columna para alinear de manera:
% -Horizontal
\newcolumntype{P}[1]{>{\centering\arraybackslash}p{#1}}
% -Vertical
\newcolumntype{M}[1]{>{\centering\arraybackslash}m{#1}}
% Indica la separacion entre las columnas de una tabla
\setlength{\tabcolsep}{10pt} % Default value: 6pt
% Indica el padding inferior y superior de las celdas de una tabla
\renewcommand{\arraystretch}{1.8} % Default value: 1
\newcommand{\finFila}{\hline}
\usepackage{graphicx}

%/////////////////////////////////////////////////////////////////
%             INICIO DOCUMENTO
%/////////////////////////////////////////////////////////////////

\begin{document}
%/////////////////////////////////////////////////////////////////
%            PAGINA 1: PROGRAMA SINTETICO
%/////////////////////////////////////////////////////////////////
%%%%%%%%%% Encabezado %%%%%%%%%%%%%%%%%%%%%%%%%%%%%%%%%%%%%%%%%%%%%%%%%%%%%%%%%%%%%%%%%%%%%%%%%%%%%%%%%%%%%%%%%%
\begin{picture}(0,0) \put(20,-60){\includegraphics[width=20mm]{Analisis/FormatoUA/ipn.png}} \end{picture}
\begin{picture}(0,0) \put(435,-60){\includegraphics[width=20mm]{Analisis/FormatoUA/ipn.png}} \end{picture}
\begin{center}
{\tab[1cm] \Large\textbf{INSTITUTO POLITÉCNICO NACIONAL}}\\
{\tab[1cm] \Large\textbf{SECRETARIA ACADÉMICA}}\\
{\tab[1cm] \large\textbf{DIRECCIÓN DE EDUCACIÓN SUPERIOR}}\\
%%%%%%%%%% Encabezado %%%%%%%%%%%%%%%%%%%%%%%%%%%%%%%%%%%%%%%%%%%%%%%%%%%%%%%%%%%%%%%%%%%%%%%%%%%%%%%%%%%%%%%%%%

\ \\ \ \\
\Centering{\Large\textbf{PROGRAMA SINTÉTICO}}\\
\end{center}
\\
\textbf{UNIDAD ACADÉMICA:} Escuela Superior de Cómputo (ESCOM)\\
\textbf{PROGRAMA ACADÉMICO:} Ingeniería en Sistemas Computacionales\\
\textbf{UNIDAD DE APRENDIZAJE:} La chona
\tab[3cm]
\textbf{SEMESTRE:} 4\\


%%%%%%%% INICIO TABLA PROPOSITO %%%%%%%%%%%%%%%%%%%%%%%%%%%%%%%%%%%%%%%%

  \begin{longtable}{|p{1.045\textwidth}|}
    \hline
    \textbf{PROPÓSITO DE LA UNIDAD DE APRENDIZAJE:}

    Elabora un sistema computacional de propósito específico con base en metodologías de Ingeniería de Software. \\

    \textbf{CONTENIDOS:}
    \begin{enumerate}[I]
    \setlength{\itemsep}{0pt}
    \setlength{\parskip}{0pt}
    \item Ingeniería de Software
\item Proceso de gestión de proyecto
\item Metodologías
\item Calidad y normas de calidad
\item Modelos de Madurez
\item Temas selectos

    \end{enumerate}

    \textbf{ORIENTACIÓN DIDÁCTICA:}

    La presente unidad se abordará a partir de la estrategia aprendizaje orientada a proyectos, el docente conducirá el curso con el método heurístico, se realizarán actividades de indagación, análisis y diseño para seleccionar la metodología adecuada al sistema de información que se quiera implementar, aplicando un enfoque de calidad; utilizando herramientas CASE para facilitar el diseño de los diagramas UML, exposiciones, prácticas y la realización de un proyecto en equipo que integre los conceptos generales así como las competencias referentes al uso de patrones de diseño y el manejo las normas calidad del software.
Las actividades que se realizarán en clase fomentarán en los estudiantes algunas técnicas, tales como: trabajo colaborativo y participativo, lluvia de ideas, organizadores gráficos, indagación documental, fichas de trabajo, exposición de temas complementarios, discusión dirigida así como la realización de un proyecto.
Para ello el docente dentro de la planeación establecerá las actividades de aprendizaje a desarrollar y los tiempos para entrega por parte del alumno; así mismo marcara los tiempos de revisión para hacer las observaciones y anotaciones para que el alumno pueda mejorar su aprendizaje, además de establecer las características del proyecto realizado. \\ 

    \textbf{EVALUACIÓN Y ACREDITACIÓN:}
    \begin{itemize}[leftmargin=*]
    \setlength{\itemsep}{0pt}
    \setlength{\parskip}{0pt}
    \item Sumativa\item rubricas de autoevaluación y coevaluación\item Saberes previos\item Equivalencia en unidad académica del IPN\item Equivalencia en extranjero\item Inscripción normal
    \end{itemize}
    \textbf{BIBLIOGRAFÍA:}
    \begin{itemize}[leftmargin=*]
    \setlength{\itemsep}{0pt}
    \setlength{\parskip}{0pt}
    \item García García, F. O. (2008) Medición y estimación del software: Técnicas y Métodos para mejorar la calidad y la productividad, México. AlfaOmega. ISBN: 9788478978588\item Piattini Piattini, M. G. , Calvo-Manzano Calvo-Manzano, J. A. (2004) Análisis y diseño de aplicaciones informáticas de gestión. Una perspectiva de Ingeniería del Software, México. AlfaOmega. ISBN: 9701509870\item Piattini Piattini, M. G. , García García, F. O. (2005) Calidad de Sistemas Informáticos, México. AlfaOmega. ISBN: 9789701512678\item Pressman Pressman, R. S. (2007) Ingeniería del software: Un enfoque Práctico, México. Mc Graw Hill. ISBN: 9701054733\item Sommerville Sommerville, I. (2008) Ingeniería de Software, España. Addison Wesley. ISBN: 9789702602064 
    \end{itemize}
    \\ \hline
  \end{longtable}
%%%%%%% FIN TABLA PROPOSITO %%%%%%%%%%%%%%%%%%%%%%%%%%%%%%%%%%%%%%%%

%%%%%%%% FIN PÁGINA 1 %%%%%%%%%%%%%%%%%%%%%%%%%%%%%%%%%%%%%%%%

%/////////////////////////////////////////////////////////////////
%            PAGINA 2: CREDITOS Y HORAS
%/////////////////////////////////////////////////////////////////
\newpage
%%%%%%%%%% Encabezado %%%%%%%%%%%%%%%%%%%%%%%%%%%%%%%%%%%%%%%%%%%%%%%%%%%%%%%%%%%%%%%%%%%%%%%%%%%%%%%%%%%%%%%%%%
\begin{picture}(0,0) \put(20,-60){\includegraphics[width=20mm]{Analisis/FormatoUA/ipn.png}} \end{picture}
\begin{picture}(0,0) \put(435,-60){\includegraphics[width=20mm]{Analisis/FormatoUA/ipn.png}} \end{picture}
\begin{center}
{\tab[1cm] \Large\textbf{INSTITUTO POLITÉCNICO NACIONAL}}\\
{\tab[1cm] \Large\textbf{SECRETARIA ACADÉMICA}}\\
{\tab[1cm] \large\textbf{DIRECCIÓN DE EDUCACIÓN SUPERIOR}}\\
\end{center}
%%%%%%%%%% Encabezado %%%%%%%%%%%%%%%%%%%%%%%%%%%%%%%%%%%%%%%%%%%%%%%%%%%%%%%%%%%%%%%%%%%%%%%%%%%%%%%%%%%%%%%%%%

%%%%%%%%%% INICIO TABLA CREDITOS %%%%%%%%%%%%%%%%%%%%%%%%%%%%%%%%%%%%%%%%%%%%%%%%%%%%%%%%%%%%%%%%%%%%%%%%%%%%%%%
\begin{table}[H]
  \begin{tabular}{|p{0.5\textwidth}|p{0.5\textwidth}|}
    \hline
    \textbf{UNIDAD ACADÉMICA:} Escuela Superior de Cómputo (ESCOM) & 
    \textbf{UNIDAD DE APRENDIZAJE:} La chona\\
    \textbf{PROGRAMA ACADÉMICO:} Ingeniería en Sistemas Computacionales & 
    \textbf{TIPO DE UNIDAD DE APRENDIZAJE:} Teórica - Práctica\\ 
    \textbf{ÁREA DE FORMACIÓN:} Profesional & 
    \textbf{VIGENCIA:} 1\\
    \textbf{MODALIDAD:} Escolarizado& 
    \textbf{SEMESTRE:} 4\\ 
    & 
    \textbf{CRÉDITOS:} 7.5 TEPIC - 7.76 SATCA\\ 
    \hline
  \end{tabular}
\end{table}
%%%%%%%%%% FIN TABLA CREDITOS %%%%%%%%%%%%%%%%%%%%%%%%%%%%%%%%%%%%%%%%%%%%%%%%%%%%%%%%%%%%%%%%%%%%%%%%%%%%%%%%

%%%%%%%%%% INICIO TABLA INTENCION %%%%%%%%%%%%%%%%%%%%%%%%%%%%%%%%%%%%%%%%%%%%%%%%%%%%%%%%%%%%%%%%%%%%%%%%%%%%%%%
\begin{table}[H]
  \begin{tabular}{|p{1.045\textwidth}|}
    \hline
    \Centering
    \textbf{INTENCIÓN EDUCATIVA:}

    \RaggedRight
    Esta unidad de aprendizaje contribuye al perfil de egresado de Ingeniería en Sistemas Computacionales, al desarrollar las habilidades de análisis y diseño de proyectos haciendo uso de software de gestión de proyectos así como herramientas CASE, además de integrar los principios de gestión de la calidad regidos por los estándares establecidos para asegurar, gestionar, auditar y certificar la calidad de procesos y productos informáticos así como también planificar y proyectar es estratégicamente (Recursos Hardware y Software, Recursos Humanos, componentes reutilizables) el desarrollo de proyectos de software. Así mismo, se dinamizan las competencias de pensamiento creativo, comunicación asertiva, trabajo colaborativo y participativo.

Se relaciona con las unidades de aprendizaje: Programación Orientada, Bases de Datos y Administración de Proyectos.\\

    \Centering
    \textbf{PROPÓSITO DE LA UNIDAD DE APRENDIZAJE:}

    \RaggedRight
    Elabora un sistema computacional de propósito específico con base en metodologías de Ingeniería de Software.\\

    \hline
  \end{tabular}
\end{table}
%%%%%%%%%% FIN TABLA INTENCION %%%%%%%%%%%%%%%%%%%%%%%%%%%%%%%%%%%%%%%%%%%%%%%%%%%%%%%%%%%%%%%%%%%%%%%%%%%%%%%

%%%%%%%%%% INICIO TIEMPOS ASIGNADOS %%%%%%%%%%%%%%%%%%%%%%%%%%%%%%%%%%%%%%%%%%%%%%%%%%%%%%%%%%%%%%%%%%%%%%%%%%%%%%%
\begin{table}[H]
  \begin{tabular}{|p{0.34\textwidth}|p{0.31\textwidth}|p{0.31\textwidth}|}
    \hline
    \centering
    \textbf{TIEMPOS ASIGNADOS}

    \raggedright
    \textbf{HORAS TEORÍA/SEMANA:} 3.0

    \textbf{HORAS PRÁCTICA/SEMANA:} 27.0
    & 
    \textbf{UNIDAD DE APRENDIZAJE DISEÑADA O REDISEÑADA POR: } S/I 
    &
    \textbf{APROBADO POR:} S/I
    S/I
    \\
    \textbf{HORAS TEORÍA/SEMESTRE:} 54.0
    \raggedright
    \textbf{HORAS PRÁCTICA/SEMESTRE:} 54.0
    & 
    \textbf{REVISADA POR:} S/I
    &\\ 
    \textbf{HORAS DE APRENDIZAJE AUTÓNOMO:} 1.5

    \textbf{HORAS TOTALES/SEMESTRE:} 81.0
    & 
    \textbf{APROBADA POR:} S/I
    S/I & 
    \textbf{AUTORIZADO Y VALIDADO POR}:\\& &\\ & \hline
    %%%%%%%%%%%%%%% AQUI VA EL QUE APRUEBA %%%%%%%%%%%%%%%%%
    \begin{center}S/I\end{center} & \hline

    %%%%%%%%%%%%%%% AQUI VA EL QUE AUTORIZA %%%%%%%%%%%%%%%%%
    \begin{center}S/I\end{center}\\ 
    \hline
  \end{tabular}
\end{table}
%%%%%%% FIN TABLA TIEMPOS ASIGNADOS %%%%%%%%%%%%%%%%%%%%%%%%%%%%%%%%%%%%%%%%

%%%%%%%% FIN PÁGINA 2 %%%%%%%%%%%%%%%%%%%%%%%%%%%%%%%%%%%%%%%%
%/////////////////////////////////////////////////////////////////
%            PAGINA 3: UNIDADES TEMATICAS Y CONTENIDOS
%/////////////////////////////////////////////////////////////////

\newpage
%%%%%%%%%% Encabezado %%%%%%%%%%%%%%%%%%%%%%%%%%%%%%%%%%%%%%%%%%%%%%%%%%%%%%%%%%%%%%%%%%%%%%%%%%%%%%%%%%%%%%%%%%
\begin{picture}(0,0) \put(20,-60){\includegraphics[width=20mm]{Analisis/FormatoUA/ipn.png}} \end{picture}
\begin{picture}(0,0) \put(435,-60){\includegraphics[width=20mm]{Analisis/FormatoUA/ipn.png}} \end{picture}
\begin{center}
{\tab[1cm] \Large\textbf{INSTITUTO POLITÉCNICO NACIONAL}}\\
{\tab[1cm] \Large\textbf{SECRETARIA ACADÉMICA}}\\
{\tab[1cm] \large\textbf{DIRECCIÓN DE EDUCACIÓN SUPERIOR}}\\
\end{center}\ \\
%%%%%%%%%% Encabezado %%%%%%%%%%%%%%%%%%%%%%%%%%%%%%%%%%%%%%%%%%%%%%%%%%%%%%%%%%%%%%%%%%%%%%%%%%%%%%%%%%%%%%%%%%

\textbf{UNIDAD DE APRENDIZAJE:} La chona
\tab[1cm]
\textbf{HOJA: } \thepage\
\tab[0.25cm]
\textbf{DE} \pageref{LastPage}\\

%%%%%%%%%% INICIO TABLA CONTENIDOS DE UNIDADES TEMATICAS %%%%%%%%%%%%%%%%%%%%%%%%%%%%%%%%%%%%%%
\begin{table}[H]
    \renewcommand{\arraystretch}{1.4}
%%%%%%%%%% INICIO NUM, NOMBRE Y COMPETENCIA DE LA UNIDAD TEMATICA %%%%%%%%%%%%%%%%%%%%%%%%%%%%%%%%%%%%%%
  \begin{tabular}{|p{0.6cm}|p{6.1cm}|p{.7cm}|p{.7cm}|p{.7cm}|p{.7cm}|p{4cm}|}
    \hline
    \multicolumn{5}{|p{8cm}}{\textbf{N$^{\circ}$ UNIDAD TEMÁTICA:} 1 } &
    \multicolumn{2}{p{6cm}|}{\textbf{NOMBRE:} Ingeniería de Software } \\
    \hline
    \multicolumn{7}{|c|}{\Centering \textbf{UNIDAD DE COMPETENCIA:}} \\
    \multicolumn{7}{|p{18.4cm}|}{\RaggedRight Compara los diferentes modelos de proceso de un sistema software con base en sus características y el tipo de proyecto a desarrollar. } \\
    \hline
    \multirow{2}{*}{\textbf{No.}} & 
    \multirow{2}{*}{\tab[1.5cm] \textbf{CONTENIDOS}} &
    \multicolumn{2}{p{2.3cm}|}{\Centering \textbf{HORAS CON DOCENTE}} &
    \multicolumn{2}{p{2.3cm}|}{\Centering \textbf{HORAS DE APRENDIZAJE AUTÓNOMO}} &
    \multirow{2}{*}{\textbf{CLAVE BIBLIOGRÁFICA}}
    \tabularnewline \cline{3-6} &&
    \multicolumn{1}{p{.7cm}|}{\Centering \textbf{T}} &
    \multicolumn{1}{p{.7cm}|}{\Centering \textbf{P}} &
    \multicolumn{1}{p{.7cm}|}{\Centering \textbf{T}} &
    \multicolumn{1}{p{.7cm}|}{\Centering \textbf{P}} &\\
    \hline
    %%%%%%%%%%%%% TEMAS %%%%%%%%%%%%%%%%%%
    1.1 & Conceptos básicos de Ingeniería de Software & 0.5 & 0.0 & 1.0 & 0.0 &Sin Clave \\ 1.2 & Atributos y características del software & 0.5 & 0.0 & 1.5 & 0.0 &Sin Clave \\ 1.3 & Importancia y aplicación del software & 0.5 & 0.0 & 1.5 & 0.0 &Sin Clave \\ 1.4 & Ciclo de vida del software & 0.5 & 0.5 & 1.5 & 0.0 &Sin Clave \\ 1.5 & Modelos de procesos & 1.0 & 1.0 & 3.5 & 0.0 &Sin Clave \\ 1.5.1 & Lineal secuencial &  &  &  &  &  \\ 1.5.2 & Cascada &  &  &  &  &  \\ 1.5.3 & Incremental &  &  &  &  &  \\ 1.5.4 & Desarrollo Rápido de Aplicaciones &  &  &  &  &  \\ 1.5.5 & Prototipos &  &  &  &  &  \\ 1.5.6 & Espiral &  &  &  &  &  \\ 
    \hline

    %%%%%%%%%%%%%% HORAS %%%%%%%%%%%%%%%
    & \RaggedRight Subtotales: &
    \Centering 3.0 &
    \Centering 1.5 &
    \Centering 9.0 &
    \Centering 0.0 &\\ 
    \hline

  \end{tabular}
\end{table}
%%%%%%%%%% FIN TABLA CONTENIDOS DE UNIDADES TEMATICAS %%%%%%%%%%%%%%%%%%%%%%%%%%%%%%%%%%%%%%

%%%%%%%%%% INICIO TABLA ESTRATEGIAS %%%%%%%%%%%%%%%%%%%%%%%%%%%%%%%%%%%%%%
\begin{table}[H]
  \begin{tabular}{|p{1.045\textwidth}|}
    \hline \Centering
    \textbf{ESTRATEGIAS DE APRENDIZAJE:}

    \RaggedRight
    Encuadre del curso.
La presente unidad se abordará a partir de la estrategia de aprendizaje orientada a proyectos y método heurístico, lo que permitirá la consolidación de las siguientes técnicas de aprendizaje: lluvia de ideas, ficha de trabajo, indagación documental, discusión dirigida, mapas conceptuales, resolución de problemas, exposición en equipo de temas complementarios, propuesta de proyecto y realización de prácticas..  \\\hline
  \end{tabular}
%%%%%%%%%% FIN TABLA ESTRATEGIAS %%%%%%%%%%%%%%%%%%%%%%%%%%%%%%%%%%%%%%

%%%%%%%%%% INICIO TABLA EVALUACION %%%%%%%%%%%%%%%%%%%%%%%%%%%%%%%%%%%%%%
  \begin{tabular}{|p{1.045\textwidth}|}
    \Centering
    \textbf{EVALUACIÓN DE LOS APRENDIZAJES:}

    \RaggedRight
    .\\\hline
  \end{tabular}
  %%%%%%%%%% FIN TABLA EVALUACION %%%%%%%%%%%%%%%%%%%%%%%%%%%%%%%%%%%%%%
\end{table}

%%%%%%%%%%% Utilizar esta pagina N veces para N unidades tematicas %%%%%%%%%%%%%%%%%%%%%%%%%%%%%%
%%%%%%%% FIN PÁGINA 3 %%%%%%%%%%%%%%%%%%%%%%%%%%%%%%%%%%%%%%%%
%/////////////////////////////////////////////////////////////////
%            PAGINA 3: UNIDADES TEMATICAS Y CONTENIDOS
%/////////////////////////////////////////////////////////////////

\newpage
%%%%%%%%%% Encabezado %%%%%%%%%%%%%%%%%%%%%%%%%%%%%%%%%%%%%%%%%%%%%%%%%%%%%%%%%%%%%%%%%%%%%%%%%%%%%%%%%%%%%%%%%%
\begin{picture}(0,0) \put(20,-60){\includegraphics[width=20mm]{Analisis/FormatoUA/ipn.png}} \end{picture}
\begin{picture}(0,0) \put(435,-60){\includegraphics[width=20mm]{Analisis/FormatoUA/ipn.png}} \end{picture}
\begin{center}
{\tab[1cm] \Large\textbf{INSTITUTO POLITÉCNICO NACIONAL}}\\
{\tab[1cm] \Large\textbf{SECRETARIA ACADÉMICA}}\\
{\tab[1cm] \large\textbf{DIRECCIÓN DE EDUCACIÓN SUPERIOR}}\\
\end{center}\ \\
%%%%%%%%%% Encabezado %%%%%%%%%%%%%%%%%%%%%%%%%%%%%%%%%%%%%%%%%%%%%%%%%%%%%%%%%%%%%%%%%%%%%%%%%%%%%%%%%%%%%%%%%%

\textbf{UNIDAD DE APRENDIZAJE:} La chona
\tab[1cm]
\textbf{HOJA: } \thepage\
\tab[0.25cm]
\textbf{DE} \pageref{LastPage}\\

%%%%%%%%%% INICIO TABLA CONTENIDOS DE UNIDADES TEMATICAS %%%%%%%%%%%%%%%%%%%%%%%%%%%%%%%%%%%%%%
\begin{table}[H]
    \renewcommand{\arraystretch}{1.4}
%%%%%%%%%% INICIO NUM, NOMBRE Y COMPETENCIA DE LA UNIDAD TEMATICA %%%%%%%%%%%%%%%%%%%%%%%%%%%%%%%%%%%%%%
  \begin{tabular}{|p{0.6cm}|p{6.1cm}|p{.7cm}|p{.7cm}|p{.7cm}|p{.7cm}|p{4cm}|}
    \hline
    \multicolumn{5}{|p{8cm}}{\textbf{N$^{\circ}$ UNIDAD TEMÁTICA:} 2 } &
    \multicolumn{2}{p{6cm}|}{\textbf{NOMBRE:} Proceso de gestión de proyecto } \\
    \hline
    \multicolumn{7}{|c|}{\Centering \textbf{UNIDAD DE COMPETENCIA:}} \\
    \multicolumn{7}{|p{18.4cm}|}{\RaggedRight Desarrolla el plan de proyecto de un sistema software con base en las técnicas de estimación aplicables en la gestión y control de los recursos, procesos y eventos. } \\
    \hline
    \multirow{2}{*}{\textbf{No.}} & 
    \multirow{2}{*}{\tab[1.5cm] \textbf{CONTENIDOS}} &
    \multicolumn{2}{p{2.3cm}|}{\Centering \textbf{HORAS CON DOCENTE}} &
    \multicolumn{2}{p{2.3cm}|}{\Centering \textbf{HORAS DE APRENDIZAJE AUTÓNOMO}} &
    \multirow{2}{*}{\textbf{CLAVE BIBLIOGRÁFICA}}
    \tabularnewline \cline{3-6} &&
    \multicolumn{1}{p{.7cm}|}{\Centering \textbf{T}} &
    \multicolumn{1}{p{.7cm}|}{\Centering \textbf{P}} &
    \multicolumn{1}{p{.7cm}|}{\Centering \textbf{T}} &
    \multicolumn{1}{p{.7cm}|}{\Centering \textbf{P}} &\\
    \hline
    %%%%%%%%%%%%% TEMAS %%%%%%%%%%%%%%%%%%
    2.1 & Ámbito de Software & 0.5 & 0.0 & 1.0 & 0.0 &Sin Clave \\ 2.2 & Estudio de factibilidad & 0.5 & 0.0 & 0.5 & 0.0 &Sin Clave \\ 2.3 & Análisis de riesgo & 0.5 & 0.5 & 1.0 & 0.0 &Sin Clave \\ 2.4 & Recursos & 0.5 & 0.0 & 0.5 & 0.0 &Sin Clave \\ 2.5 & Estimación & 0.5 & 1.5 & 2.5 & 0.0 &Sin Clave \\ 2.5.1 & Métricas &  &  &  &  &  \\ 2.5.2 & Modelos de estimación &  &  &  &  &  \\ 2.6 & Planificación del proyecto & 0.5 & 0.5 & 1.0 & 0.0 &Sin Clave \\ 2.6.1 & Calendario de actividades &  &  &  &  &  \\ 2.6.2 & Diagrama de Gantt &  &  &  &  &  \\ 2.6.3 & Diagrama de Pert &  &  &  &  &  \\ 2.7 & Supervisión y control del plan de proyecto & 0.0 & 0.5 & 1.0 & 0.0 &Sin Clave \\ 
    \hline

    %%%%%%%%%%%%%% HORAS %%%%%%%%%%%%%%%
    & \RaggedRight Subtotales: &
    \Centering 3.0 &
    \Centering 3.0 &
    \Centering 7.5 &
    \Centering 0.0 &\\ 
    \hline

  \end{tabular}
\end{table}
%%%%%%%%%% FIN TABLA CONTENIDOS DE UNIDADES TEMATICAS %%%%%%%%%%%%%%%%%%%%%%%%%%%%%%%%%%%%%%

%%%%%%%%%% INICIO TABLA ESTRATEGIAS %%%%%%%%%%%%%%%%%%%%%%%%%%%%%%%%%%%%%%
\begin{table}[H]
  \begin{tabular}{|p{1.045\textwidth}|}
    \hline \Centering
    \textbf{ESTRATEGIAS DE APRENDIZAJE:}

    \RaggedRight
    La presente unidad se abordará a partir de la estrategia aprendizaje orientada a proyectos y método heurístico, lo que permitirá la consolidación de las siguientes técnicas de aprendizaje: indagación documental, ficha de trabajo, discusión dirigida, cuadro de comparaciones , líneas de tiempo, diagrama de Gantt y diagramas de pert, exposición en equipo de temas complementarios, avance del proyecto y realización de prácticas..  \\\hline
  \end{tabular}
%%%%%%%%%% FIN TABLA ESTRATEGIAS %%%%%%%%%%%%%%%%%%%%%%%%%%%%%%%%%%%%%%

%%%%%%%%%% INICIO TABLA EVALUACION %%%%%%%%%%%%%%%%%%%%%%%%%%%%%%%%%%%%%%
  \begin{tabular}{|p{1.045\textwidth}|}
    \Centering
    \textbf{EVALUACIÓN DE LOS APRENDIZAJES:}

    \RaggedRight
    .\\\hline
  \end{tabular}
  %%%%%%%%%% FIN TABLA EVALUACION %%%%%%%%%%%%%%%%%%%%%%%%%%%%%%%%%%%%%%
\end{table}

%%%%%%%%%%% Utilizar esta pagina N veces para N unidades tematicas %%%%%%%%%%%%%%%%%%%%%%%%%%%%%%
%%%%%%%% FIN PÁGINA 3 %%%%%%%%%%%%%%%%%%%%%%%%%%%%%%%%%%%%%%%%
%/////////////////////////////////////////////////////////////////
%            PAGINA 3: UNIDADES TEMATICAS Y CONTENIDOS
%/////////////////////////////////////////////////////////////////

\newpage
%%%%%%%%%% Encabezado %%%%%%%%%%%%%%%%%%%%%%%%%%%%%%%%%%%%%%%%%%%%%%%%%%%%%%%%%%%%%%%%%%%%%%%%%%%%%%%%%%%%%%%%%%
\begin{picture}(0,0) \put(20,-60){\includegraphics[width=20mm]{Analisis/FormatoUA/ipn.png}} \end{picture}
\begin{picture}(0,0) \put(435,-60){\includegraphics[width=20mm]{Analisis/FormatoUA/ipn.png}} \end{picture}
\begin{center}
{\tab[1cm] \Large\textbf{INSTITUTO POLITÉCNICO NACIONAL}}\\
{\tab[1cm] \Large\textbf{SECRETARIA ACADÉMICA}}\\
{\tab[1cm] \large\textbf{DIRECCIÓN DE EDUCACIÓN SUPERIOR}}\\
\end{center}\ \\
%%%%%%%%%% Encabezado %%%%%%%%%%%%%%%%%%%%%%%%%%%%%%%%%%%%%%%%%%%%%%%%%%%%%%%%%%%%%%%%%%%%%%%%%%%%%%%%%%%%%%%%%%

\textbf{UNIDAD DE APRENDIZAJE:} La chona
\tab[1cm]
\textbf{HOJA: } \thepage\
\tab[0.25cm]
\textbf{DE} \pageref{LastPage}\\

%%%%%%%%%% INICIO TABLA CONTENIDOS DE UNIDADES TEMATICAS %%%%%%%%%%%%%%%%%%%%%%%%%%%%%%%%%%%%%%
\begin{table}[H]
    \renewcommand{\arraystretch}{1.4}
%%%%%%%%%% INICIO NUM, NOMBRE Y COMPETENCIA DE LA UNIDAD TEMATICA %%%%%%%%%%%%%%%%%%%%%%%%%%%%%%%%%%%%%%
  \begin{tabular}{|p{0.6cm}|p{6.1cm}|p{.7cm}|p{.7cm}|p{.7cm}|p{.7cm}|p{4cm}|}
    \hline
    \multicolumn{5}{|p{8cm}}{\textbf{N$^{\circ}$ UNIDAD TEMÁTICA:} 3 } &
    \multicolumn{2}{p{6cm}|}{\textbf{NOMBRE:} Metodologías } \\
    \hline
    \multicolumn{7}{|c|}{\Centering \textbf{UNIDAD DE COMPETENCIA:}} \\
    \multicolumn{7}{|p{18.4cm}|}{\RaggedRight Clasifica las diferentes metodologías con base en las etapas que las conforman y los resultados obtenidos en cada una de ellas. } \\
    \hline
    \multirow{2}{*}{\textbf{No.}} & 
    \multirow{2}{*}{\tab[1.5cm] \textbf{CONTENIDOS}} &
    \multicolumn{2}{p{2.3cm}|}{\Centering \textbf{HORAS CON DOCENTE}} &
    \multicolumn{2}{p{2.3cm}|}{\Centering \textbf{HORAS DE APRENDIZAJE AUTÓNOMO}} &
    \multirow{2}{*}{\textbf{CLAVE BIBLIOGRÁFICA}}
    \tabularnewline \cline{3-6} &&
    \multicolumn{1}{p{.7cm}|}{\Centering \textbf{T}} &
    \multicolumn{1}{p{.7cm}|}{\Centering \textbf{P}} &
    \multicolumn{1}{p{.7cm}|}{\Centering \textbf{T}} &
    \multicolumn{1}{p{.7cm}|}{\Centering \textbf{P}} &\\
    \hline
    %%%%%%%%%%%%% TEMAS %%%%%%%%%%%%%%%%%%
    3.1 & Metodologías estructuradas & 1.0 & 0.5 & 2.5 & 0.0 &Sin Clave \\ 3.1.1 & Merisse &  &  &  &  &  \\ 3.1.2 & Yourdon &  &  &  &  &  \\ 3.1.3 & Gane-Sarson &  &  &  &  &  \\ 3.2 & Metodologías Orientadas a Objetos & 1.0 & 0.5 & 3.5 & 0.0 &Sin Clave \\ 3.2.1 & OMT &  &  &  &  &  \\ 3.2.2 & Ingeniería de Software Orientado a Objetos (Jacobson) &  &  &  &  &  \\ 3.2.3 & Proceso Unificado &  &  &  &  &  \\ 3.2.4 & Proceso Unificado de Rational &  &  &  &  &  \\ 3.3 & Proceso Unificado de Rational & 1.0 & 0.5 & 3.0 & 0.0 &Sin Clave \\ 3.3.1 & Programación Extrema &  &  &  &  &  \\ 3.3.2 & SCRUM &  &  &  &  &  \\ 3.3.3 & Crystal &  &  &  &  &  \\ 
    \hline

    %%%%%%%%%%%%%% HORAS %%%%%%%%%%%%%%%
    & \RaggedRight Subtotales: &
    \Centering 3.0 &
    \Centering 1.5 &
    \Centering 9.0 &
    \Centering 0.0 &\\ 
    \hline

  \end{tabular}
\end{table}
%%%%%%%%%% FIN TABLA CONTENIDOS DE UNIDADES TEMATICAS %%%%%%%%%%%%%%%%%%%%%%%%%%%%%%%%%%%%%%

%%%%%%%%%% INICIO TABLA ESTRATEGIAS %%%%%%%%%%%%%%%%%%%%%%%%%%%%%%%%%%%%%%
\begin{table}[H]
  \begin{tabular}{|p{1.045\textwidth}|}
    \hline \Centering
    \textbf{ESTRATEGIAS DE APRENDIZAJE:}

    \RaggedRight
    La presente unidad se abordará a partir de la estrategia aprendizaje orientada a proyectos y método heurístico, lo que permitirá la consolidación de las siguientes técnicas de aprendizaje: indagación documental, ficha de trabajo, discusión dirigida, cuadro de comparaciones , mapas conceptuales, exposición en equipo de temas complementarios, avance del proyecto y realización de prácticas..  \\\hline
  \end{tabular}
%%%%%%%%%% FIN TABLA ESTRATEGIAS %%%%%%%%%%%%%%%%%%%%%%%%%%%%%%%%%%%%%%

%%%%%%%%%% INICIO TABLA EVALUACION %%%%%%%%%%%%%%%%%%%%%%%%%%%%%%%%%%%%%%
  \begin{tabular}{|p{1.045\textwidth}|}
    \Centering
    \textbf{EVALUACIÓN DE LOS APRENDIZAJES:}

    \RaggedRight
    .\\\hline
  \end{tabular}
  %%%%%%%%%% FIN TABLA EVALUACION %%%%%%%%%%%%%%%%%%%%%%%%%%%%%%%%%%%%%%
\end{table}

%%%%%%%%%%% Utilizar esta pagina N veces para N unidades tematicas %%%%%%%%%%%%%%%%%%%%%%%%%%%%%%
%%%%%%%% FIN PÁGINA 3 %%%%%%%%%%%%%%%%%%%%%%%%%%%%%%%%%%%%%%%%
%/////////////////////////////////////////////////////////////////
%            PAGINA 3: UNIDADES TEMATICAS Y CONTENIDOS
%/////////////////////////////////////////////////////////////////

\newpage
%%%%%%%%%% Encabezado %%%%%%%%%%%%%%%%%%%%%%%%%%%%%%%%%%%%%%%%%%%%%%%%%%%%%%%%%%%%%%%%%%%%%%%%%%%%%%%%%%%%%%%%%%
\begin{picture}(0,0) \put(20,-60){\includegraphics[width=20mm]{Analisis/FormatoUA/ipn.png}} \end{picture}
\begin{picture}(0,0) \put(435,-60){\includegraphics[width=20mm]{Analisis/FormatoUA/ipn.png}} \end{picture}
\begin{center}
{\tab[1cm] \Large\textbf{INSTITUTO POLITÉCNICO NACIONAL}}\\
{\tab[1cm] \Large\textbf{SECRETARIA ACADÉMICA}}\\
{\tab[1cm] \large\textbf{DIRECCIÓN DE EDUCACIÓN SUPERIOR}}\\
\end{center}\ \\
%%%%%%%%%% Encabezado %%%%%%%%%%%%%%%%%%%%%%%%%%%%%%%%%%%%%%%%%%%%%%%%%%%%%%%%%%%%%%%%%%%%%%%%%%%%%%%%%%%%%%%%%%

\textbf{UNIDAD DE APRENDIZAJE:} La chona
\tab[1cm]
\textbf{HOJA: } \thepage\
\tab[0.25cm]
\textbf{DE} \pageref{LastPage}\\

%%%%%%%%%% INICIO TABLA CONTENIDOS DE UNIDADES TEMATICAS %%%%%%%%%%%%%%%%%%%%%%%%%%%%%%%%%%%%%%
\begin{table}[H]
    \renewcommand{\arraystretch}{1.4}
%%%%%%%%%% INICIO NUM, NOMBRE Y COMPETENCIA DE LA UNIDAD TEMATICA %%%%%%%%%%%%%%%%%%%%%%%%%%%%%%%%%%%%%%
  \begin{tabular}{|p{0.6cm}|p{6.1cm}|p{.7cm}|p{.7cm}|p{.7cm}|p{.7cm}|p{4cm}|}
    \hline
    \multicolumn{5}{|p{8cm}}{\textbf{N$^{\circ}$ UNIDAD TEMÁTICA:} 4 } &
    \multicolumn{2}{p{6cm}|}{\textbf{NOMBRE:} Calidad y normas de calidad } \\
    \hline
    \multicolumn{7}{|c|}{\Centering \textbf{UNIDAD DE COMPETENCIA:}} \\
    \multicolumn{7}{|p{18.4cm}|}{\RaggedRight Realiza un sistema de información de calidad con base en las diferentes normas enfocadas a los productos software. } \\
    \hline
    \multirow{2}{*}{\textbf{No.}} & 
    \multirow{2}{*}{\tab[1.5cm] \textbf{CONTENIDOS}} &
    \multicolumn{2}{p{2.3cm}|}{\Centering \textbf{HORAS CON DOCENTE}} &
    \multicolumn{2}{p{2.3cm}|}{\Centering \textbf{HORAS DE APRENDIZAJE AUTÓNOMO}} &
    \multirow{2}{*}{\textbf{CLAVE BIBLIOGRÁFICA}}
    \tabularnewline \cline{3-6} &&
    \multicolumn{1}{p{.7cm}|}{\Centering \textbf{T}} &
    \multicolumn{1}{p{.7cm}|}{\Centering \textbf{P}} &
    \multicolumn{1}{p{.7cm}|}{\Centering \textbf{T}} &
    \multicolumn{1}{p{.7cm}|}{\Centering \textbf{P}} &\\
    \hline
    %%%%%%%%%%%%% TEMAS %%%%%%%%%%%%%%%%%%
    4.1 & Conceptos de la calidad & 1.0 & 0.0 & 1.5 & 0.0 &Sin Clave \\ 4.2 & Calidad de sistemas de información & 0.5 & 0.5 & 3.0 & 0.0 &Sin Clave \\ 4.3 & Calidad del producto software & 1.0 & 0.5 & 1.5 & 0.0 &Sin Clave \\ 4.4 & Modelos y normas de calidad & 1.0 & 0.0 & 3.5 & 0.0 &Sin Clave \\ 4.4.1 & ISO 9000 &  &  &  &  &  \\ 4.4.2 & ISO 25000 &  &  &  &  &  \\ 4.4.3 & IEEE Std 1061-1998 &  &  &  &  &  \\ 4.4.4 & ISO/IEC 15939 &  &  &  &  &  \\ 
    \hline

    %%%%%%%%%%%%%% HORAS %%%%%%%%%%%%%%%
    & \RaggedRight Subtotales: &
    \Centering 3.5 &
    \Centering 1.0 &
    \Centering 9.5 &
    \Centering 0.0 &\\ 
    \hline

  \end{tabular}
\end{table}
%%%%%%%%%% FIN TABLA CONTENIDOS DE UNIDADES TEMATICAS %%%%%%%%%%%%%%%%%%%%%%%%%%%%%%%%%%%%%%

%%%%%%%%%% INICIO TABLA ESTRATEGIAS %%%%%%%%%%%%%%%%%%%%%%%%%%%%%%%%%%%%%%
\begin{table}[H]
  \begin{tabular}{|p{1.045\textwidth}|}
    \hline \Centering
    \textbf{ESTRATEGIAS DE APRENDIZAJE:}

    \RaggedRight
    La presente unidad se abordará a partir de la estrategia aprendizaje orientada a proyectos y método heurístico, lo que permitirá la consolidación de las siguientes técnicas de aprendizaje: indagación documental, ficha de trabajo, discusión dirigida, cuadro de comparaciones, documentación del proyecto (diagramas UML, Pert y Gantt, estudio de factibilidad, análisis de riesgos, modelo relacional de la base datos y diccionario de datos), exposición en equipo de temas complementarios y realización de prácticas..  \\\hline
  \end{tabular}
%%%%%%%%%% FIN TABLA ESTRATEGIAS %%%%%%%%%%%%%%%%%%%%%%%%%%%%%%%%%%%%%%

%%%%%%%%%% INICIO TABLA EVALUACION %%%%%%%%%%%%%%%%%%%%%%%%%%%%%%%%%%%%%%
  \begin{tabular}{|p{1.045\textwidth}|}
    \Centering
    \textbf{EVALUACIÓN DE LOS APRENDIZAJES:}

    \RaggedRight
    .\\\hline
  \end{tabular}
  %%%%%%%%%% FIN TABLA EVALUACION %%%%%%%%%%%%%%%%%%%%%%%%%%%%%%%%%%%%%%
\end{table}

%%%%%%%%%%% Utilizar esta pagina N veces para N unidades tematicas %%%%%%%%%%%%%%%%%%%%%%%%%%%%%%
%%%%%%%% FIN PÁGINA 3 %%%%%%%%%%%%%%%%%%%%%%%%%%%%%%%%%%%%%%%%
%/////////////////////////////////////////////////////////////////
%            PAGINA 3: UNIDADES TEMATICAS Y CONTENIDOS
%/////////////////////////////////////////////////////////////////

\newpage
%%%%%%%%%% Encabezado %%%%%%%%%%%%%%%%%%%%%%%%%%%%%%%%%%%%%%%%%%%%%%%%%%%%%%%%%%%%%%%%%%%%%%%%%%%%%%%%%%%%%%%%%%
\begin{picture}(0,0) \put(20,-60){\includegraphics[width=20mm]{Analisis/FormatoUA/ipn.png}} \end{picture}
\begin{picture}(0,0) \put(435,-60){\includegraphics[width=20mm]{Analisis/FormatoUA/ipn.png}} \end{picture}
\begin{center}
{\tab[1cm] \Large\textbf{INSTITUTO POLITÉCNICO NACIONAL}}\\
{\tab[1cm] \Large\textbf{SECRETARIA ACADÉMICA}}\\
{\tab[1cm] \large\textbf{DIRECCIÓN DE EDUCACIÓN SUPERIOR}}\\
\end{center}\ \\
%%%%%%%%%% Encabezado %%%%%%%%%%%%%%%%%%%%%%%%%%%%%%%%%%%%%%%%%%%%%%%%%%%%%%%%%%%%%%%%%%%%%%%%%%%%%%%%%%%%%%%%%%

\textbf{UNIDAD DE APRENDIZAJE:} La chona
\tab[1cm]
\textbf{HOJA: } \thepage\
\tab[0.25cm]
\textbf{DE} \pageref{LastPage}\\

%%%%%%%%%% INICIO TABLA CONTENIDOS DE UNIDADES TEMATICAS %%%%%%%%%%%%%%%%%%%%%%%%%%%%%%%%%%%%%%
\begin{table}[H]
    \renewcommand{\arraystretch}{1.4}
%%%%%%%%%% INICIO NUM, NOMBRE Y COMPETENCIA DE LA UNIDAD TEMATICA %%%%%%%%%%%%%%%%%%%%%%%%%%%%%%%%%%%%%%
  \begin{tabular}{|p{0.6cm}|p{6.1cm}|p{.7cm}|p{.7cm}|p{.7cm}|p{.7cm}|p{4cm}|}
    \hline
    \multicolumn{5}{|p{8cm}}{\textbf{N$^{\circ}$ UNIDAD TEMÁTICA:} 5 } &
    \multicolumn{2}{p{6cm}|}{\textbf{NOMBRE:} Modelos de Madurez } \\
    \hline
    \multicolumn{7}{|c|}{\Centering \textbf{UNIDAD DE COMPETENCIA:}} \\
    \multicolumn{7}{|p{18.4cm}|}{\RaggedRight Aplica los modelos de madurez, evaluación y mejora de procesos en el desarrollo de software con base en el control de calidad, los productos finales y los niveles definidos por dichos procesos. } \\
    \hline
    \multirow{2}{*}{\textbf{No.}} & 
    \multirow{2}{*}{\tab[1.5cm] \textbf{CONTENIDOS}} &
    \multicolumn{2}{p{2.3cm}|}{\Centering \textbf{HORAS CON DOCENTE}} &
    \multicolumn{2}{p{2.3cm}|}{\Centering \textbf{HORAS DE APRENDIZAJE AUTÓNOMO}} &
    \multirow{2}{*}{\textbf{CLAVE BIBLIOGRÁFICA}}
    \tabularnewline \cline{3-6} &&
    \multicolumn{1}{p{.7cm}|}{\Centering \textbf{T}} &
    \multicolumn{1}{p{.7cm}|}{\Centering \textbf{P}} &
    \multicolumn{1}{p{.7cm}|}{\Centering \textbf{T}} &
    \multicolumn{1}{p{.7cm}|}{\Centering \textbf{P}} &\\
    \hline
    %%%%%%%%%%%%% TEMAS %%%%%%%%%%%%%%%%%%
    5.1 & Introducción & 0.5 & 0.0 & 1.0 & 0.0 &Sin Clave \\ 5.2 & Proceso de Software Personal (PSP) & 0.5 & 0.0 & 1.5 & 0.0 &Sin Clave \\ 5.3 & Proceso de Software de Equipo (TSP) & 0.5 & 0.0 & 2.0 & 0.0 &Sin Clave \\ 5.4 & Modelo de Capacidad de Madurez (CMM) & 0.5 & 0.5 & 2.0 & 0.0 &Sin Clave \\ 5.5 & Modelo de Capacidad de Madurez Integrado (CMMI) & 0.5 & 0.5 & 2.5 & 0.0 &Sin Clave \\ 5.6 & MoProSoft & 0.5 & 0.0 & 2.5 & 0.0 &Sin Clave \\ 
    \hline

    %%%%%%%%%%%%%% HORAS %%%%%%%%%%%%%%%
    & \RaggedRight Subtotales: &
    \Centering 3.0 &
    \Centering 1.0 &
    \Centering 11.5 &
    \Centering 0.0 &\\ 
    \hline

  \end{tabular}
\end{table}
%%%%%%%%%% FIN TABLA CONTENIDOS DE UNIDADES TEMATICAS %%%%%%%%%%%%%%%%%%%%%%%%%%%%%%%%%%%%%%

%%%%%%%%%% INICIO TABLA ESTRATEGIAS %%%%%%%%%%%%%%%%%%%%%%%%%%%%%%%%%%%%%%
\begin{table}[H]
  \begin{tabular}{|p{1.045\textwidth}|}
    \hline \Centering
    \textbf{ESTRATEGIAS DE APRENDIZAJE:}

    \RaggedRight
    La presente unidad se abordará a partir de la estrategia aprendizaje orientada a proyectos y método heurístico, lo que permitirá la consolidación de las siguientes técnicas de aprendizaje: indagación documental, ficha de trabajo, discusión dirigida, cuadro de comparaciones, programa de cómputo, exposición en equipo de temas complementarios, conclusión del proyecto y realización de prácticas..  \\\hline
  \end{tabular}
%%%%%%%%%% FIN TABLA ESTRATEGIAS %%%%%%%%%%%%%%%%%%%%%%%%%%%%%%%%%%%%%%

%%%%%%%%%% INICIO TABLA EVALUACION %%%%%%%%%%%%%%%%%%%%%%%%%%%%%%%%%%%%%%
  \begin{tabular}{|p{1.045\textwidth}|}
    \Centering
    \textbf{EVALUACIÓN DE LOS APRENDIZAJES:}

    \RaggedRight
    .\\\hline
  \end{tabular}
  %%%%%%%%%% FIN TABLA EVALUACION %%%%%%%%%%%%%%%%%%%%%%%%%%%%%%%%%%%%%%
\end{table}

%%%%%%%%%%% Utilizar esta pagina N veces para N unidades tematicas %%%%%%%%%%%%%%%%%%%%%%%%%%%%%%
%%%%%%%% FIN PÁGINA 3 %%%%%%%%%%%%%%%%%%%%%%%%%%%%%%%%%%%%%%%%
%/////////////////////////////////////////////////////////////////
%            PAGINA 3: UNIDADES TEMATICAS Y CONTENIDOS
%/////////////////////////////////////////////////////////////////

\newpage
%%%%%%%%%% Encabezado %%%%%%%%%%%%%%%%%%%%%%%%%%%%%%%%%%%%%%%%%%%%%%%%%%%%%%%%%%%%%%%%%%%%%%%%%%%%%%%%%%%%%%%%%%
\begin{picture}(0,0) \put(20,-60){\includegraphics[width=20mm]{Analisis/FormatoUA/ipn.png}} \end{picture}
\begin{picture}(0,0) \put(435,-60){\includegraphics[width=20mm]{Analisis/FormatoUA/ipn.png}} \end{picture}
\begin{center}
{\tab[1cm] \Large\textbf{INSTITUTO POLITÉCNICO NACIONAL}}\\
{\tab[1cm] \Large\textbf{SECRETARIA ACADÉMICA}}\\
{\tab[1cm] \large\textbf{DIRECCIÓN DE EDUCACIÓN SUPERIOR}}\\
\end{center}\ \\
%%%%%%%%%% Encabezado %%%%%%%%%%%%%%%%%%%%%%%%%%%%%%%%%%%%%%%%%%%%%%%%%%%%%%%%%%%%%%%%%%%%%%%%%%%%%%%%%%%%%%%%%%

\textbf{UNIDAD DE APRENDIZAJE:} La chona
\tab[1cm]
\textbf{HOJA: } \thepage\
\tab[0.25cm]
\textbf{DE} \pageref{LastPage}\\

%%%%%%%%%% INICIO TABLA CONTENIDOS DE UNIDADES TEMATICAS %%%%%%%%%%%%%%%%%%%%%%%%%%%%%%%%%%%%%%
\begin{table}[H]
    \renewcommand{\arraystretch}{1.4}
%%%%%%%%%% INICIO NUM, NOMBRE Y COMPETENCIA DE LA UNIDAD TEMATICA %%%%%%%%%%%%%%%%%%%%%%%%%%%%%%%%%%%%%%
  \begin{tabular}{|p{0.6cm}|p{6.1cm}|p{.7cm}|p{.7cm}|p{.7cm}|p{.7cm}|p{4cm}|}
    \hline
    \multicolumn{5}{|p{8cm}}{\textbf{N$^{\circ}$ UNIDAD TEMÁTICA:} 6 } &
    \multicolumn{2}{p{6cm}|}{\textbf{NOMBRE:} Temas selectos } \\
    \hline
    \multicolumn{7}{|c|}{\Centering \textbf{UNIDAD DE COMPETENCIA:}} \\
    \multicolumn{7}{|p{18.4cm}|}{\RaggedRight Reestructura los proyectos software con base en nuevos requerimientos y los conceptos fundamentales de la Ingeniería de Software. } \\
    \hline
    \multirow{2}{*}{\textbf{No.}} & 
    \multirow{2}{*}{\tab[1.5cm] \textbf{CONTENIDOS}} &
    \multicolumn{2}{p{2.3cm}|}{\Centering \textbf{HORAS CON DOCENTE}} &
    \multicolumn{2}{p{2.3cm}|}{\Centering \textbf{HORAS DE APRENDIZAJE AUTÓNOMO}} &
    \multirow{2}{*}{\textbf{CLAVE BIBLIOGRÁFICA}}
    \tabularnewline \cline{3-6} &&
    \multicolumn{1}{p{.7cm}|}{\Centering \textbf{T}} &
    \multicolumn{1}{p{.7cm}|}{\Centering \textbf{P}} &
    \multicolumn{1}{p{.7cm}|}{\Centering \textbf{T}} &
    \multicolumn{1}{p{.7cm}|}{\Centering \textbf{P}} &\\
    \hline
    %%%%%%%%%%%%% TEMAS %%%%%%%%%%%%%%%%%%
    6.1 & Herramientas Case & 1.0 & 0.5 & 3.5 & 0.0 &Sin Clave \\ 6.2 & Ingeniería Web & 1.0 & 0.0 & 2.5 & 0.0 &Sin Clave \\ 6.2.1 & Conceptos y Planeación &  &  &  &  &  \\ 6.2.2 & Modelado y Pruebas &  &  &  &  &  \\ 6.3 & Reingeniería & 1.0 & 0.5 & 3.5 & 0.0 &Sin Clave \\ 6.3.1 & Procesos de Negocio &  &  &  &  &  \\ 6.3.2 & Del Software &  &  &  &  &  \\ 6.3.3 & Reestructuración &  &  &  &  &  \\ 6.3.4 & Ingeniería Inversa &  &  &  &  &  \\ 6.3.5 & Ingeniería Directa &  &  &  &  &  \\ 
    \hline

    %%%%%%%%%%%%%% HORAS %%%%%%%%%%%%%%%
    & \RaggedRight Subtotales: &
    \Centering 3.0 &
    \Centering 1.0 &
    \Centering 9.5 &
    \Centering 0.0 &\\ 
    \hline

  \end{tabular}
\end{table}
%%%%%%%%%% FIN TABLA CONTENIDOS DE UNIDADES TEMATICAS %%%%%%%%%%%%%%%%%%%%%%%%%%%%%%%%%%%%%%

%%%%%%%%%% INICIO TABLA ESTRATEGIAS %%%%%%%%%%%%%%%%%%%%%%%%%%%%%%%%%%%%%%
\begin{table}[H]
  \begin{tabular}{|p{1.045\textwidth}|}
    \hline \Centering
    \textbf{ESTRATEGIAS DE APRENDIZAJE:}

    \RaggedRight
    La presente unidad se abordará a partir de la estrategia aprendizaje orientada a proyectos y método heurístico, lo que permitirá la consolidación de las siguientes técnicas de aprendizaje: indagación documental, ficha de trabajo, discusión dirigida, cuadro de comparaciones, programa de cómputo, exposición en equipo de temas complementarios, proyecto reestructurado y realización de prácticas..  \\\hline
  \end{tabular}
%%%%%%%%%% FIN TABLA ESTRATEGIAS %%%%%%%%%%%%%%%%%%%%%%%%%%%%%%%%%%%%%%

%%%%%%%%%% INICIO TABLA EVALUACION %%%%%%%%%%%%%%%%%%%%%%%%%%%%%%%%%%%%%%
  \begin{tabular}{|p{1.045\textwidth}|}
    \Centering
    \textbf{EVALUACIÓN DE LOS APRENDIZAJES:}

    \RaggedRight
    .\\\hline
  \end{tabular}
  %%%%%%%%%% FIN TABLA EVALUACION %%%%%%%%%%%%%%%%%%%%%%%%%%%%%%%%%%%%%%
\end{table}

%%%%%%%%%%% Utilizar esta pagina N veces para N unidades tematicas %%%%%%%%%%%%%%%%%%%%%%%%%%%%%%
%%%%%%%% FIN PÁGINA 3 %%%%%%%%%%%%%%%%%%%%%%%%%%%%%%%%%%%%%%%%
%/////////////////////////////////////////////////////////////////
%            PAGINA 4: RELACION PRACTICAS
%/////////////////////////////////////////////////////////////////

\newpage
%%%%%%%%%% Encabezado %%%%%%%%%%%%%%%%%%%%%%%%%%%%%%%%%%%%%%%%%%%%%%%%%%%%%%%%%%%%%%%%%%%%%%%%%%%%%%%%%%%%%%%%%%
\begin{picture}(0,0) \put(20,-60){\includegraphics[width=20mm]{Analisis/FormatoUA/ipn.png}} \end{picture}
\begin{picture}(0,0) \put(435,-60){\includegraphics[width=20mm]{Analisis/FormatoUA/ipn.png}} \end{picture}
\begin{center}
{\tab[1cm] \Large\textbf{INSTITUTO POLITÉCNICO NACIONAL}}\\
{\tab[1cm] \Large\textbf{SECRETARIA ACADÉMICA}}\\
{\tab[1cm] \large\textbf{DIRECCIÓN DE EDUCACIÓN SUPERIOR}}\\
\end{center}\ \\
%%%%%%%%%% Encabezado %%%%%%%%%%%%%%%%%%%%%%%%%%%%%%%%%%%%%%%%%%%%%%%%%%%%%%%%%%%%%%%%%%%%%%%%%%%%%%%%%%%%%%%%%%

\textbf{UNIDAD DE APRENDIZAJE:} La chona
\tab[1cm]
\textbf{HOJA: } \thepage\
\tab[0.25cm]
\textbf{DE } \pageref{LastPage}\\
\begin{center}
\Centering{\Large\textbf{RELACIÓN DE PRÁCTICAS}}
\end{center}
%%%%%%%%%% INICIO TABLA PRACTICAS %%%%%%%%%%%%%%%%%%%%%%%%%%%%%%%%%%%%%%%%%%%%%%%%%%%%%%%%%%%%%%%%%%%%%%%%%%%%%%%%%%%%%%%%%%
\begin{table}[H]
  \begin{tabular}{|p{0.1\textwidth}|p{0.315\textwidth}|p{0.12\textwidth}|p{0.12\textwidth}|p{0.22\textwidth}|}
    \hline
    \Centering\textbf{PRÁCTICA No.} & \Centering\textbf{NOMBRE DE LA PRÁCTICA} & \Centering\textbf{UNIDADES TEMÁTICAS} & \Centering\textbf{DURACIÓN} & \Centering\textbf{LUGAR DE REALIZACIÓN}\\
    \hline 
%%%%%%%%%% NUM PRACTICA, NOMBRE, UNIDADES TEMATICAS, DURACION, LUGAR %%%%%%%%%%%%%%%%%%%%%%%%%%%%%%%%%%%%%%%%%%%%
    1&Modelos de procesos&1&4.5&Salas de cómputo de la Escuela.\\2&Proceso de gestión de proyecto&2&4.5&Salas de cómputo de la Escuela.\\3&Uso de las diferentes Metodologías&3&4.5&Salas de cómputo de la Escuela.\\4&Listas de verificación usados en las Normas de Calidad&4&4.5&Salas de cómputo de la Escuela.\\5&Desarrollar un ejemplo para la demostración de una herramienta CASE&5&4.5&Salas de cómputo de la Escuela.\\6&Realizar una demostración del proceso de Ingeniería inversa utilizando una herramienta CASE Específica&6&4.5&Salas de cómputo de la Escuela.\\
%%%%%%%%%% TOTAL DE HRS %%%%%%%%%%%%%%%%%%%%%%%%%%%%%%%%%%%%%%%%%%%%%%%%%%%%%%%%%%%%%%%%%%%%%%%%%%%%%%%%%%%%%%%%%%
    \hline &&\Centering\textbf{TOTAL DE HORAS}& 27.0 &\\\hline 
  \end{tabular}
%%%%%%%%%% FIN TABLA PRACTICAS %%%%%%%%%%%%%%%%%%%%%%%%%%%%%%%%%%%%%%%%%%%%%%%%%%%%%%%%%%%%%%%%%%%%%%%%%%%%%%%%%%%%%%%%%%
%%%%%%%%%% INICIO TABLA EVALUACION %%%%%%%%%%%%%%%%%%%%%%%%%%%%%%%%%%%%%%%%%%%%%%%%%%%%%%%%%%%%%%%%%%%%%%%%%%%%%%%%%%%%%%%%%%
  \begin{tabular}{|p{1.045\textwidth}|}
    \textbf{EVALUACIÓN Y ACREDITACIÓN:}

    Las prácticas aportan el 15\% de la calificación de la unidad temática I.\newline Las prácticas aportan el 10\% de la calificación de la unidad temática II.\newline Las prácticas aportan el 20\% de la calificación de la unidad temática III.\newline Las prácticas aportan el 20\% de la calificación de la unidad temática IV.\newline Las prácticas aportan el 20\% de la calificación de la unidad temática V.\newline Las prácticas aportan el 10\% de la calificación de la unidad temática V.\newline Las prácticas se consideran requisito indispensable para acreditar esta unidad de aprendizaje.\newline \\\hline
  \end{tabular}
%%%%%%%%%% FIN TABLA EVALUACION %%%%%%%%%%%%%%%%%%%%%%%%%%%%%%%%%%%%%%%%%%%%%%%%%%%%%%%%%%%%%%%%%%%%%%%%%%%%%%%%%%%%%%%%%%
\end{table}


%%%%%%%% FIN PÁGINA 4 %%%%%%%%%%%%%%%%%%%%%%%%%%%%%%%%%%%%%%%%

%/////////////////////////////////////////////////////////////////
%            PAGINA 5: EVALUACION
%/////////////////////////////////////////////////////////////////
\newpage
%%%%%%%%%% Encabezado %%%%%%%%%%%%%%%%%%%%%%%%%%%%%%%%%%%%%%%%%%%%%%%%%%%%%%%%%%%%%%%%%%%%%%%%%%%%%%%%%%%%%%%%%%
\begin{picture}(0,0) \put(20,-60){\includegraphics[width=20mm]{Analisis/FormatoUA/ipn.png}} \end{picture}
\begin{picture}(0,0) \put(435,-60){\includegraphics[width=20mm]{Analisis/FormatoUA/ipn.png}} \end{picture}
\begin{center}
{\tab[1cm] \Large\textbf{INSTITUTO POLITÉCNICO NACIONAL}}\\
{\tab[1cm] \Large\textbf{SECRETARIA ACADÉMICA}}\\
{\tab[1cm] \large\textbf{DIRECCIÓN DE EDUCACIÓN SUPERIOR}}\\
\end{center}\\\
%%%%%%%%%% Encabezado %%%%%%%%%%%%%%%%%%%%%%%%%%%%%%%%%%%%%%%%%%%%%%%%%%%%%%%%%%%%%%%%%%%%%%%%%%%%%%%%%%%%%%%%%%

\textbf{UNIDAD DE APRENDIZAJE:} La chona
\tab[1cm]
\textbf{HOJA: } \thepage\
\tab[0.25cm]
\textbf{DE } \pageref{LastPage}\\

%%%%%%%%%% INICIO TABLA EVALUACION %%%%%%%%%%%%%%%%%%%%%%%%%%%%%%%%%%%%%%%%%%%%%%%%%%%%%%%%%%%%%%%%%%%%%%%%
\begin{table}[H]

  \begin{tabular}{|p{0.1\textwidth}|p{0.1\textwidth}|p{0.76\textwidth}|}
    \hline

    \textbf{PERIODO} & \textbf{UNIDAD} & \textbf{PROCEDIMIENTO DE EVALUACIÓN}\\\hline

    %%%%%%%%%% PERIODO, UNIDAD, EVALUACION %%%%%%%%%%%%%%%%%%%%%%%%%%%%%%%%%%%%%%
    

    \hline
  \end{tabular}
\end{table}
%%%%%%%%%% FIN TABLA EVALUACION %%%%%%%%%%%%%%%%%%%%%%%%%%%%%%%%%%%%%%%%%%%%%%%%%%%%%%%%%%%%%%%%%%%%%%%%

%%%%%%%% FIN PÁGINA 5 %%%%%%%%%%%%%%%%%%%%%%%%%%%%%%%%%%%%%%%%

%/////////////////////////////////////////////////////////////////
%            PAGINA 6: BIBLIOGRAFIA
%/////////////////////////////////////////////////////////////////
\newpage
%%%%%%%%%% Encabezado %%%%%%%%%%%%%%%%%%%%%%%%%%%%%%%%%%%%%%%%%%%%%%%%%%%%%%%%%%%%%%%%%%%%%%%%%%%%%%%%%%%%%%%%%%
\begin{picture}(0,0) \put(20,-60){\includegraphics[width=20mm]{Analisis/FormatoUA/ipn.png}} \end{picture}
\begin{picture}(0,0) \put(435,-60){\includegraphics[width=20mm]{Analisis/FormatoUA/ipn.png}} \end{picture}
\begin{center}
{\tab[1cm] \Large\textbf{INSTITUTO POLITÉCNICO NACIONAL}}\\
{\tab[1cm] \Large\textbf{SECRETARIA ACADÉMICA}}\\
{\tab[1cm] \large\textbf{DIRECCIÓN DE EDUCACIÓN SUPERIOR}}\\
\end{center}\ \\
%%%%%%%%%% Encabezado %%%%%%%%%%%%%%%%%%%%%%%%%%%%%%%%%%%%%%%%%%%%%%%%%%%%%%%%%%%%%%%%%%%%%%%%%%%%%%%%%%%%%%%%%%

\textbf{UNIDAD DE APRENDIZAJE:} La chona
\tab[1cm]
\textbf{HOJA: } \thepage
\tab[0.25cm]
\textbf{DE } \pageref{LastPage}\\

%%%%%%%%%% INICIO TABLA BIBLIO %%%%%%%%%%%%%%%%%%%%%%%%%%%%%%%%%%%%%%%%%%%%%%%%%%%%%%%%%%%%%%%%%%%%%%%%%%%%%%%%%%%%%%%%
\begin{table}[H]
  \begin{tabular}{|p{0.1\textwidth}|p{0.1\textwidth}|p{0.1\textwidth}|p{0.62\textwidth}|}
    \hline

    \textbf{CLAVE} & \textbf{B} & \textbf{C} & \textbf{BIBLIOGRAFÍA}\\\hline
%%%%%%%%%% CLAVE, B, X, BIBLIOGRAFIA %%%%%%%%%%%%%%%%%%%%%%%%%%%%%%%%%%%%%%%%%%%%%%%%%%%%%%%%%%%%%%%%%%%%%%%%%%%%%%%%%%%%%%
    1&X &  &García García, F. O. (2008) Medición y estimación del software: Técnicas y Métodos para mejorar la calidad y la productividad, México. AlfaOmega. ISBN: 9788478978588\\ 2& & X &Humphrey Humphrey, W. S. (2005) PSP: A Self-Improvement Process for Software Engineers, Estados Unidos de América. Addison Wesley. ISBN: 9780321305497\\ 3& & X &Kimmel Kimmel, P. (2006) Manual de UML, España. Mc Graw Hill. ISBN: 9789701058992\\ 4&X &  &Ila Ila, J. A. (2006) UML 2, España. Anaya. ISBN: 9788441520332\\ 5& & X &Palacio Palacio, J. (2010) El día a día en los proyectos Software, España. Lulu.com. ISBN: 9781847531339\\ 6&X &  &Piattini Piattini, M. G. , Calvo-Manzano Calvo-Manzano, J. A. (2004) Análisis y diseño de aplicaciones informáticas de gestión. Una perspectiva de Ingeniería del Software, México. AlfaOmega. ISBN: 9701509870\\ 7&X &  &Piattini Piattini, M. G. , García García, F. O. (2005) Calidad de Sistemas Informáticos, México. AlfaOmega. ISBN: 9789701512678\\ 8&X &  &Pressman Pressman, R. S. (2007) Ingeniería del software: Un enfoque Práctico, México. Mc Graw Hill. ISBN: 9701054733\\ 9&X &  &Priolo Priolo, S. (2009) Métodos Ágiles, Argentina. Ed.Users. ISBN: 9789871347971\\ 10& & X &Schach Schach, S. R. (2005) Análisis y diseño orientado a objetos con UML y el proceso unificado, España. Mc Graw Hill. ISBN: 9789701049822\\ 11&X &  &Sommerville Sommerville, I. (2008) Ingeniería de Software, España. Addison Wesley. ISBN: 9789702602064\\ 12& & X &Whitten Whitten, J. L. (2008) Análisis de sistemas: diseño y métodos, España. Mc Graw Hill. ISBN: 9789701066140\\ 

    \hline
  \end{tabular}
\end{table}
%%%%%%%%%% FIN TABLA BIBLIO %%%%%%%%%%%%%%%%%%%%%%%%%%%%%%%%%%%%%%%%%%%%%%%%%%%%%%%%%%%%%%%%%%%%%%%%%%%%%%%%%%%%%%%%

%%%%%%%% FIN PÁGINA 6 %%%%%%%%%%%%%%%%%%%%%%%%%%%%%%%%%%%%%%%%

%/////////////////////////////////////////////////////////////////
%            PAGINA 7: PERFIL DOCENTE
%/////////////////////////////////////////////////////////////////
\newpage
%%%%%%%%%% Encabezado %%%%%%%%%%%%%%%%%%%%%%%%%%%%%%%%%%%%%%%%%%%%%%%%%%%%%%%%%%%%%%%%%%%%%%%%%%%%%%%%%%%%%%%%%%
\begin{picture}(0,0) \put(20,-60){\includegraphics[width=20mm]{Analisis/FormatoUA/ipn.png}} \end{picture}
\begin{picture}(0,0) \put(435,-60){\includegraphics[width=20mm]{Analisis/FormatoUA/ipn.png}} \end{picture}
\begin{center}
{\tab[1cm] \Large\textbf{INSTITUTO POLITÉCNICO NACIONAL}}\\
{\tab[1cm] \Large\textbf{SECRETARIA ACADÉMICA}}\\
{\tab[1cm] \large\textbf{DIRECCIÓN DE EDUCACIÓN SUPERIOR}}\\
%%%%%%%%%% Encabezado %%%%%%%%%%%%%%%%%%%%%%%%%%%%%%%%%%%%%%%%%%%%%%%%%%%%%%%%%%%%%%%%%%%%%%%%%%%%%%%%%%%%%%%%%%
\ \\ \ \\
\Centering{\Large\textbf{PERFIL DOCENTE POR UNIDAD DE APRENDIZAJE}}\\
\end{center}
\\
\begin{enumerate}
  %%%%%%%%%% DATOS GENERALES %%%%%%%%%%%%%%%%%%%%%%%%%%%%%%%%%%%%%%%%%%%%%%%%%%%%%%%%%%%%%%%%%%%%%%%%%%%%%%%%%%%%%%%%%%
    \item \textbf{DATOS GENERALES}
    \\ \ \\
    \textbf{UNIDAD ACADÉMICA:} Escuela Superior de Cómputo (ESCOM)\\ \ \\
    \textbf{PROGRAMA ACADÉMICO:} Ingeniería en Sistemas Computacionales
    \tab[1cm]
    \textbf{SEMESTRE:} 4\\

%%%%%%%%%% INICIO AREA DE FORMACION %%%%%%%%%%%%%%%%%%%%%%%%%%%%%%%%%%%%%%%%%%%%%%%%%%%%%%%%%%%%%%%%%%%%%%%%%%%%%%%%%%%%%%%
    \begin{tabular}{|p{0.15\textwidth}|p{0.15\textwidth}|p{0.15\textwidth}|p{0.15\textwidth}|p{0.15\textwidth}|}
      \hline
      \textbf{ÁREA DE FORMACIÓN} & \textbf{Institucional} &\textbf{Científica}
      \textbf{Básica} & \textbf{Profesional}  & \textbf{Terminal y de Integración}
      \\\hline
    \end{tabular}\\
%%%%%%%%%% FIN AREA DE FORMACION %%%%%%%%%%%%%%%%%%%%%%%%%%%%%%%%%%%%%%%%%%%%%%%%%%%%%%%%%%%%%%%%%%%%%%%%%%%%%%%%%%%%%%%%%%

    \textbf{ACADEMIA:} Ingeniería de Software
    \tab[1cm]
    \textbf{UNIDAD DE APRENDIZAJE:} La chona\\ \ \\
    \textbf{ESPECIALIDAD Y NIVEL ACADÉMICO REQUERIDO:} Experiencia de dos años en en el análisis de Sistemas de Información (Líder de Proyecto), Experiencia de dos años en el manejo de grupos y en el trabajo colaborativo. o Experiencia de un año como Docente de Nivel Superior.\\

    \item \textbf{PROPÓSITO DE LA UNIDAD DE APRENDIZAJE}
    %%%%%%%%%% AQUI VA EL PROPOSITO %%%%%%%%%%%%%%%%%%%%%%%%%%%%%%%%%%%%%%%%%%%%%%%%%%%%%%%%%%%%%%%%%%%%%%%%%%%%%%%%%%%%%%%%%%
    Elabora un sistema computacional de propósito específico con base en metodologías de Ingeniería de Software.
    \item \textbf{PERFIL DOCENTE}\\
    
    %%%%%%%%%% INICIO TABLA CONOCIMIENTOS %%%%%%%%%%%%%%%%%%%%%%%%%%%%%%%%%%%%%%%%%%%%%%%%%%%%%%%%%%%%%%%%%%%%%%%%%%%%%%%%%%%%
    \begin{tabular}{|p{0.2\textwidth}|p{0.2\textwidth}|p{0.2\textwidth}|p{0.2\textwidth}|}
      \hline
      \textbf{CONOCIMIENTOS} & \textbf{EXPERIENCIA PROFESIONAL} &\textbf{COMPETENCIAS DOCENTES} & \textbf{ACTITUDES}
      \\\hline
      %%%%%%%%%% CONOCIMIENTOS %%%%%%%%%%%%%%%%%%%%%%%%%%%%%%%%%%%%%%%%%%%%%%%%%%%%%%%%
      \begin{itemize}[leftmargin=*]
      \setlength{\itemsep}{0pt}
      \setlength{\parskip}{0pt}
      \item Capacidad para el manejo de grupos\item Fluidez verbal de ideas\item Capacidad de traspasar conocimientos\item Manejo de grupos y trabajo colaborativo\item Manejo de estrategias para fomentar el aprendizaje autónomo en el alumno\item Manejo de estrategias didácticas centradas en el aprendizaje\item Aplicación del MEI
      \end{itemize}
      & 
       %%%%%%%%%% EXPERIENCIA %%%%%%%%%%%%%%%%%%%%%%%%%%%%%%%%%%%%%%%%%%%%%%%%%%%%%%%%
      \begin{itemize}[leftmargin=*]
      \setlength{\itemsep}{0pt}
      \setlength{\parskip}{0pt}
      \item Experiencia de dos años en en el análisis de Sistemas de Información (Líder de Proyecto)\item Experiencia de dos años en el manejo de grupos y en el trabajo colaborativo.\item Experiencia de un año como Docente de Nivel Superior.
      \end{itemize}
      &
       %%%%%%%%%% COMPETENCIAS %%%%%%%%%%%%%%%%%%%%%%%%%%%%%%%%%%%%%%%%%%%%%%%%%%%%%%%%
      \begin{itemize}[leftmargin=*]
      \setlength{\itemsep}{0pt}
      \setlength{\parskip}{0pt}
      \item Capacidad para el manejo de grupos\item Fluidez verbal de ideas\item Capacidad de traspasar conocimientos\item Manejo de grupos y trabajo colaborativo\item Manejo de estrategias para fomentar el aprendizaje autónomo en el alumno\item Manejo de estrategias didácticas centradas en el aprendizaje\item Aplicación del MEI
      \end{itemize}
      & 
       %%%%%%%%%% ACTITUDES %%%%%%%%%%%%%%%%%%%%%%%%%%%%%%%%%%%%%%%%%%%%%%%%%%%%%%%%
      \begin{itemize}[leftmargin=*]
      \setlength{\itemsep}{0pt}
      \setlength{\parskip}{0pt}
      \item Responsable\item Honesto\item Respetuoso\item Tolerante\item Asertivo\item Colaborativo\item Participativo
      \end{itemize}
      \\\hline
    \end{tabular}
    %%%%%%%%%% FIN TABLA CONOCIMIENTOS %%%%%%%%%%%%%%%%%%%%%%%%%%%%%%%%%%%%%%%%%%%%%%%%%%%%%%%%%%%%%%%%%%%%%%%%%%%%%%%%%%%%
\end{enumerate}

 %%%%%%%%%% INICIO TABLA DIRECTIVOS %%%%%%%%%%%%%%%%%%%%%%%%%%%%%%%%%%%%%%%%%%%%%%%%%%%%%%%%
  \begin{tabular}{p{0.28\textwidth}p{0.28\textwidth}p{0.28\textwidth}}

      \centering
      \textbf{ELABORÓ} 
      &
      \centering
      \textbf{REVISÓ} 
      &
      \centering
      \textbf{AUTORIZÓ}\\
      &&&\\

      \centering
      \hline 
       %%%%%%%%%% ELABORO %%%%%%%%%%%%%%%%%%%%%%%%%%%%%%%%%%%%%%%%%%%%%%%%%%%%%%%%
      S/I
      &
      \centering
      \hline 
       %%%%%%%%%% REVISO %%%%%%%%%%%%%%%%%%%%%%%%%%%%%%%%%%%%%%%%%%%%%%%%%%%%%%%%
      S/I
      &
      \centering
      \hline 
       %%%%%%%%%% AUTORIZO %%%%%%%%%%%%%%%%%%%%%%%%%%%%%%%%%%%%%%%%%%%%%%%%%%%%%%%%
       S/I
  \end{tabular}
   %%%%%%%%%% FIN TABLA DIRECTIVOS %%%%%%%%%%%%%%%%%%%%%%%%%%%%%%%%%%%%%%%%%%%%%%%%%%%%%%%%
%%%%%%%% FIN PÁGINA 7 %%%%%%%%%%%%%%%%%%%%%%%%%%%%%%%%%%%%%%%%
\end{document}
